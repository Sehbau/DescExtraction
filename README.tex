Here we run the program 'dscx'. It creates the descriptor output for an image. Two
executables are available:

dscx.exe    compiled under Windows 10.
dscx	    compiled under Ubuntu ...

For documentation see descExtr.pdf. In the following, only an overview is given.

The program takes the image formats jpg and png. It writes various types of features
to files.

Directories: 

- Imgs	 contains example images for immediate probing
- Desc   demonstrates what the output files look like (as generated by dscx) 
- UtilMb contains Matlab scripts to load some of the output files

Program 'dscx' takes two file names as input (arguments):
   1) an image path
   2) an output file path. This path needs to include an existing directory name.

For example: with images in 'Imgs', and the output directory 'Desc', run image 'img1'
as follows:

   dscx Imgs\img1.jpg Desc\img1	   (windows)
   dscx Imgs/img1.jpg Desc/img1	   (linux)

The output file path must contain a valid path ('Desc/' in this case), as the
program checks for slashes, otherwise it aborts. The program will write the
following files into directory 'Desc':

    - img1.Bbox	    bounding boxes of regions
    - img1.CntEpt   contour endpoints of ridge, river and edge segments 
    - img1.dsc	    descriptor vectors (attributes)
    - img1.hst	    descriptor histograms

You can load the first two files into Matlab as demonstrated in directory UtilMb,
see script exampleLoad.m for explanations.

For more info see: descExtr.pdf


